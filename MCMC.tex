\documentclass{article}

\usepackage{Sweave}
\begin{document}
\Sconcordance{concordance:MCMC.tex:MCMC.rnw:%
1 2 1 1 0 138 1}


\section{Bayesian Inference}
\begin{verbatim}
Probability Theory is used to quantify uncertainties.

Probability Distributions are associated with uncertain
measurements.

The specification of probability distributions to the
uncertain deterministic relations between some of them
defines a statistical model.

y = percentage of people who remembered having watched the ad
x = advertising expenditure
/pi = awareness probability

Simplest link between advertising expenditure and awareness
is given by the linear relation

\pi = \alpha + \Beta x


\end{verbatim}

\section{Markov Chain Monte Carlo}

\section{General Overview}
\subsection{Stochastic simulation}
\subsection{Bayesian inference}
\subsection{Approximate methods of inference}
\subsection{Markov Chains}
\subsection{Gibbs Sampling}
\subsection{Metropolis-Hastings algorithms}
\begin{verbatim}
#############################################################################################################
#
# Example 6.1 (pages 192-3) – Figures 6.1 (page 194)  - Random walk Metropolis-Hastings algorithm
#
# y[i] : velocity of an enzymatic reacton (in counts/min/min) as a function of 
#        substrate concentration
# x[i] : (in ppm) where the enzyme has been treated with Puromycin
# 
#
# Referencia: Carlin and Louis (1998) - Bayes and Empirical 
# Bayes Methods for Data Analysis - Chapman & Hall/CRC
# 
# paginas: 233-234.
#
##############################################################################################################
#
# Author : Hedibert Freitas Lopes
#          Graduate School of Business
#          University of Chicago
#          5807 South Woodlawn Avenue
#          Chicago, Illinois, 60637
#          Email : hlopes@ChicagoGSB.edu
#
###############################################################################################################
mu      = function(th,x){gamma+alpha*x/(th+x)}
like    = function(th){prod(dnorm(y,mu(th,x),sqrt(tau2)))}
loglike = function(th){sum(dnorm(y,mu(th,x),sqrt(tau2),log=TRUE))}
logpost = function(th){dnorm(th,theta0,Ctheta,log=TRUE)+loglike(th)}
n       = 12
x       = c(0.02,0.02,0.06,0.06,0.11,0.11,0.22,0.22,0.56,0.56,1.10,1.10)
y       = c(76,47,97,107,123,139,159,152,191,201,207,200)
gamma   = 50
alpha   = 170
tau2    = 126
theta0  = 0.0
Ctheta  = 100
thetas  = seq(0.08,0.20,length=1000)
likes   = NULL
for (i in 1:1000)
  likes = c(likes,like(thetas[i]))
thetamle = thetas[likes==max(likes)]
xs       = seq(min(x),max(x),length=1000)
mus      = NULL
for (i in 1:1000)
  mus = c(mus,mu(thetamle,xs[i]))

par(mfrow=c(1,2))
plot(thetas,likes,xlab="theta",ylab="likelihood",type="l")
plot(x,y)
lines(xs,mus,col=2)

##############################################################################
# Random Walk Metropolis-Hastings
##############################################################################
set.seed(192796)
Vtheta = 0.1
theta  = 0.4
burnin = 0
M      = 1000
sample = theta
ind    = 0
for (iter in 1:(burnin+M-1)){
  th1 = rnorm(1,theta,Vtheta)
  prob = exp(logpost(th1)-logpost(theta))
  if (runif(1) < prob){
    theta = th1
    ind   = ind + 1
  }
  sample = c(sample,theta)
}
sample = sample[(burnin+1):length(sample)]
acceptancerate = length(unique(sample))/(M+1)
mean=mean(sample)
sd=sqrt(var(sample))
L=quantile(sample,0.025)
U=quantile(sample,0.975)
c(mean,sd,L,U)


# Figure 6.1
# ----------
par(mfrow=c(1,1))
plot(sample,xlab="iteration",ylab=expression(theta),main="",type="l",cex=3,axes=F)
axis(1)
axis(2)


\end{verbatim}

\section{MCMCpack}
\begin{verbatim}
install.packages("MCMCpack")
data(swiss)
posterior1 <- MCMCregress(Fertility ~ Agriculture + Examination + Eduction + Catholic + Infant.Mortality,data=swiss)
summary(posterior1)
\end{verbatim}
source:

\emph{Markov Chain Monte Carlo}. Dani Gamerman and Hedibert F. Lopes.
ChapmanHall CRC Boca Raton Florida. 2006.


\end{document}
