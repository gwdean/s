% Created 2013-09-03 Tue 12:51
\documentclass[11pt]{article}
\usepackage[utf8]{inputenc}
\usepackage[T1]{fontenc}
\usepackage{graphicx}
\usepackage{longtable}
\usepackage{float}
\usepackage{wrapfig}
\usepackage{soul}
\usepackage{amssymb}
\usepackage{hyperref}


\title{Cosmic}
\author{Greg}
\date{03 September 2013}

\begin{document}

\maketitle

\setcounter{tocdepth}{3}
\tableofcontents
\vspace*{1cm}
\section{Cosmic}
\label{sec-1}

Notes on Helier Robinson's \emph{Cosmic Coincidences}.
\href{http://arxiv-web3.library.cornell.edu/abs/1111.4562}{http://arxiv-web3.library.cornell.edu/abs/1111.4562}
\subsection{Abstract}
\label{sec-1.1}

   Arising out of an attempt at a new foundations of mathematics, in which relations are more primitive than sets, and out of the theoretical physicists' concept of underlying causes of empirical phenomena, the idea of a purely mathematical possible world (of underlying causes) is developed. It is shown that at least one, and at most one, possible world is actual, and that the one that is actual is the best (as in the philosophy of Leibniz), and therefore requires the cosmic coincidences to exist. This best is actual necessarily because of having a higher level top relation than any other possible world, and because of this top relation possessing the property of intrinsic necessary existence. 
\subsection{The Problem of Cosmic Coincidences}
\label{sec-1.2}

The \emph{Problem of Cosmic Coincidences} refers to the improbability
of certain parameters in theoretical physics having the values
that they do. Following Lee Smolin (Smolin, 1997) Robinson lists
four solutions to this problem:

\begin{itemize}
\item Anthropic explanation
\item Fine-tuning explanation
\item One consistent mathematical system
\item Universes emerge from black holes
\end{itemize}
For Robinson, none of these options are satisfactory, so he posits
a fifth explanation, which derives from the work of GW Leibniz.
Leibniz claims that the ``actual world exists necessarily because
it is the best of all possibles.'' The world that Leibniz refers
to is not the world of empirical phenomena that we experience, but
rather the world of underlying causes of these empirical phenomena.

So for Robinson, there are two worlds, the \emph{EMPIRICAL WORLD} and 
the \emph{UNDERLYING WORLD}, and it is crucial for the argument that 
we are clear about which of these worlds that we are talking about.

The core of the argument rests on the concept of \emph{NECESSARY EXISTENCE}
or \emph{ACTUALITY}. According to Robinson: ``If this concept is neither 
self-contradictory nor implies a contradiction, then whatever has
this necessary existence must exist in at least one among all 
possible worlds, making that one possible world actual.''

\subsection{Relational Philosophy of Mathematics}
\label{sec-1.3}

Robinson's argument rests on a philosophy of mathematics
based on relations rather than sets. According to Robinson,
in set theory, the defintion of relations ``involves a 
vicious circle'': the definition of relations as subsets
of Cartesian products presupposes the relation of set
membership and the relations of logical argument forms.

Since relations (arguably) are more important than sets
in mathematics, and sets can be defined easily in terms
of relations, relations should be made fundamental in
mathematics.

Once relations are made fundamental, we can distinguish
between three kinds of defined sets:

\begin{itemize}
\item Intensional sets
\item Extensional sets
\item Nominal sets
\end{itemize}
The \emph{INTENSION} of a set (if it has one) is all those properties
that all and only its members possess.

The \emph{EXTENSION} of a set is the totality of its members.

A \emph{NOMINAL SET} is a set that has a name or a description but
otherwise does not exist.

\subsection{Possible and Actual}
\label{sec-1.4}

\subsection{Summary of Argument}
\label{sec-1.5}

\subsubsection{Overview of what has been said so far}
\label{sec-1.5.1}

\begin{itemize}
\item At most one possible world can be actual
\item At least one possible world must be actual
\item At most one possible world containing a relation possessing INE can exist
\item At least one relation possessing INE must exist
\end{itemize}
The only way in which all these four conditions can occur together
is if this relation possessing INE emerges as the top relation of a
possible world having a higher top level than any other possible
world, which makes the relation possessing INE unique among all
possible worlds. The possible world possessing this unique relation
must be the best of all possibles \textbf{G}. This unique relation must
be \textbf{T}, possessing the emergent property of INE. Thus:


\subsection{Bibliographic Information}
\label{sec-1.6}


\end{document}
