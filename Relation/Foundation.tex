% Created 2013-08-30 Fri 14:52
\documentclass[11pt]{article}
\usepackage[utf8]{inputenc}
\usepackage[T1]{fontenc}
\usepackage{graphicx}
\usepackage{longtable}
\usepackage{float}
\usepackage{wrapfig}
\usepackage{soul}
\usepackage{amssymb}
\usepackage{hyperref}


\title{Foundation}
\author{Greg}
\date{30 August 2013}

\begin{document}

\maketitle

\setcounter{tocdepth}{3}
\tableofcontents
\vspace*{1cm}
\section{Relations as a Foundation for Mathematics}
\label{sec-1}

\subsection{Primitive Concepts}
\label{sec-1.1}

Two primitives:
\begin{itemize}
\item \textbf{RELATION}
\item \textbf{POSSIBILITY}
\end{itemize}
Secondary primitive concepts will be various particular relations.

Possibility is of two kinds:
\begin{itemize}
\item Intrinsic
\item Extrinsic
\end{itemize}
Possibility is covered in the section on the Ontological Argument

Relations have three essential characteristics
\begin{itemize}
\item They are \textbf{SIMPLE}
\item They have both \textbf{INTRINSIC} and \textbf{EXTRINSIC PROPERTIES}
\item One of the intrinsic properties is an \textbf{ADICITY}
\end{itemize}
The intrinsic properties of a relation are what determines the
\textbf{KIND} of the relation that it is, and the extrinsic properties
determine the \textbf{INSTANCE} of that kind.

Extrinsic properties are defined by means of skew-separability:

A is \textbf{SKEW-SEPARABLE} from B if A can exist without B,
but B cannot exist without A.

If a relation R is skew-separable from another relation, S,
the R is a \textbf{LOWER EXTRINSIC PROPERTY} of S, and S is an
\textbf{UPPER EXTRINSIC PROPERTY\} of R. And a property of a relation
is an \textbf{INTRINSIC PROPERTY} of that relation if the relation
and the property are inseparable.

A primitive relation which exists independently of any
mind is a \textbf{REAL RELATION}.

A primitive relation which exists within a mind is an
\textbf{IDEAL RELATION\}, also called an \textbf{ABSTRACT IDEA}.

A relation which is either real or ideal is a \textbf{GENUINE RELATION},
also called an \textbf{INTENSIONAL RELATION}. 

A subset of a Cartesian product is an \textbf{EXTENSIONAL RELATION}

A grammatical form of words that indicates a relation is
called a \textbf{NOMINAL RELATION}; if it does not refer to either
a genuine relation or an extensional relation then it is
called a \textbf{PURELY NOMINAL RELATION}.
\subsection{Sets}
\label{sec-1.2}

An \textbf{INTENSIONAL SET} is a plurality united by a relation.

A \textbf{MEMBER} of an intensional set is any one element of the
unified plurality; the relation between a member and its 
set is the relation of \textbf{SET-MEMBERSHIP}.

The \textbf{EXTENSION} of an intensional set is its plurality.

The \textbf{INTENSION} of an intensional set is the commonality of
its plurality: those properties, intrinsic or extrinsic,
possessed by all and only the members of the set; as such
the intension is an extrinsic property of each member.

The \textbf{FUNCTION EVERY} is the necessity relation which has
intensions as its arguments and intensional sets as its
values; its inverse is the \textbf{FUNCTION ANY}.


\subsection{Three Kinds of Mathematical Meaning}
\label{sec-1.3}

\subsection{Mathematical Reasoning}
\label{sec-1.4}

A \textbf{RELATIONAL MEANING} is the relation that unifies an intensional set.

A \textbf{COMPOUNDABLE RELATION} is any kind of relation that has one or more
properties which are also possessed by the relation which unites a set
of instances of that kind. A \textbf{COMPOUND RELATION} is an extrinsic 
intensional set of compoundable relations, whose relational meaning
is of a kind similar to, or a superintension of, the kind of its
members.
 
\subsection{Foundations}
\label{sec-1.5}

\subsection{Some Theorems}
\label{sec-1.6}


\end{document}
