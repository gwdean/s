% Created 2013-08-31 Sat 17:08
\documentclass[11pt]{article}
\usepackage[utf8]{inputenc}
\usepackage[T1]{fontenc}
\usepackage{graphicx}
\usepackage{longtable}
\usepackage{float}
\usepackage{wrapfig}
\usepackage{soul}
\usepackage{amssymb}
\usepackage{hyperref}


\title{Notes on the Helier Robinson's Philosophy of Mathematics}
\author{Greg Dean}
\date{2013-08-31 Sat}

\begin{document}

\maketitle

\setcounter{tocdepth}{3}
\tableofcontents
\vspace*{1cm}

\section{Intensional Philosophy of Mathematics}
\label{sec-1}

\subsection{Preface}
\label{sec-1.1}

A \textbf{PURELY INTENSIONAL MATHEMATICS} has two merits:

\begin{itemize}
\item Axiom Generosity
\item Freedom from Paradox
\end{itemize}
Main Problems in the Philosophy of Mathematics

\begin{enumerate}
\item What is it that determines mathematical inferences,
\end{enumerate}
as if there were a separate Platonic reality which
mathematics describes but cannot alter?

\begin{enumerate}
\item Why are axiom sets in mathematics so rich in consequences,
\end{enumerate}
so generous with theorems?

\begin{enumerate}
\item What is the explanation of both the power and the beauty
\end{enumerate}
of mathematics, each of which puts mathematics in an entirely
different category from all other languages?

\begin{enumerate}
\item What is the difference between mathematical discovery and
\end{enumerate}
mathematical invention? And, whichever is involved, what is
the origin of mathematical novelty?

\begin{enumerate}
\item Why is mathematics so effective in describing the world?
\item Given our distinction between intensional, extensional,
\end{enumerate}
and nominal mathematics, what is the extent of intensional
mathematics.

Here are some answers to the questions above:

\emph{Mathematical Inferences\} are based on relations of necessity,
or extrinsic properties that include necessity.

\emph{Axiom Generosity\} is due to cascading emergence of relations.

Mathematics is \emph{Effective In Describing the World} because 
reality is relational and mathematics is our language of 
relations. 

To explain the power, beauty and novelties of mathematics
we need a concept called \emph{hekergy}.
\subsection{Hekergy}
\label{sec-1.2}

\subsection{The Ontological Argument}
\label{sec-1.3}


\end{document}
